\documentclass[11pt]{article}
\usepackage[margin=1in]{geometry}
\usepackage{amsmath, amssymb}
\usepackage{enumitem}
\usepackage{tikz}
\usepackage{booktabs}
\usepackage{graphicx}

\title{HW \#4}
\author{Theo Tarr}
\date{February 26, 2026}

\begin{document}
\maketitle

\section*{1. A Chemical Bond}

\textbf{(a)} With no interference, probability = sum of the individual probabilities:
\[
P = |\psi_1|^2 + |\psi_2|^2 = (0.35)^2 + (0.22)^2 = 0.1225 + 0.0484 = \boxed{0.1709}
\]

\textbf{(b)} With constructive interference, add the amplitudes before squaring:
\[
P = |\psi_1 + \psi_2|^2 = (0.35 + 0.22)^2 = (0.57)^2 = \boxed{0.3249}
\]

\textbf{(c)} The wave probability (0.32) is almost double the probability if there was no interference (0.17). This increased electron density between the two positive nuclei creates a covalent bond that holds them together. Because of the constructive interference, there is a higher probability of finding the electron between the two nuclei, which lowers the overall energy of the system and forms a stable bond.

\begin{figure}
\centering
\includegraphics[width=0.5\linewidth]{0CA4DC78-8301-44DB-974D-70536716A0A0_1_201_a.jpeg}
\caption{Constructive interference increases electron density between two nuclei}
\label{fig:placeholder}
\end{figure}


\section*{2. Hydrogen Atom}

\textbf{(a)} The longest wavelength has least energy, which is the $E_3 \to E_2$ transition.

\[
E_3 = -1.5\ \text{eV}
\]
\[
E_2 = -3.4\ \text{eV}
\]

\[
\Delta E = E_3 - E_2 = -1.5 - (-3.4) = 1.9\ \text{eV}
\]
\[
\lambda = \frac{hc}{\Delta E} = \frac{1241\ \text{eV}\cdot\text{nm}}{1.9\ \text{eV}} \approx \boxed{653\ \text{nm (a red wavelength)}}
\]

\textbf{(b)}
\[
E_1 = -13.6\ \text{eV}
\]
\[
E_5 = -0.54\ \text{eV}
\]
\[
\Delta E = E_5 - E_1 = -0.54 - (-13.6) = \boxed{13.06\ \text{eV}}
\]

\textbf{(c)}
\[
\frac{P(E_5)}{P(E_1)} = e^{-\Delta E / k_B T} = \frac{1}{1000}
\]
\[
-\frac{13.056}{k_B T} = \ln(\frac{1}{1000}) = -6.908
\]
\[
k_B T = \frac{13.056}{6.908} = 1.890 \text{ eV}
\]
\[
T = \frac{1.890 \text{ eV}}{8.617 \times 10^{-5} \text{ eV/K}} = \boxed{21{,}900 \text{ K}}
\]

\section*{3. Chemical Reactions}

\textbf{(a)} The diagram shows potential energy versus position with wave functions for each state (A, B, and transition).
\begin{figure}
    \centering
    \includegraphics[width=0.5\linewidth]{807F8F1B-1CA5-4785-B7BD-C9D1F89A6F56_1_201_a.jpeg}
    \caption{Wave functions for a double-well: state A, state B, and transition state.}
    \label{fig:placeholder}
\end{figure}

Each well has standing-wave states where $n=1$ is the lowest energy states in each well. More nodes will have higher energy (in a box, $E_n=\frac{n^2 h^2}{8mL^2}$), so $n=1$ will have the lowest energy. Thus, there must be a standing wave for state A and another for state B. To go from A $\to$ B, the molecule must pass through a transition-state standing wave that is at even higher energy.

\textbf{(b)}
\[
\text{rate}=\nu e^{-E_b/(k_BT)},\quad \text{time}=\frac{1}{\text{rate}},\quad
\nu\approx10^{13}\,\text{s}^{-1},\quad k_BT=(8.617\times10^{-5})(310)=0.02671\,\text{eV}
\]

\[
E_b=0.1\,\text{eV}:\quad
\text{rate}=10^{13}e^{-0.1/0.02671}=2.4\times10^{11}\,\text{s}^{-1},\quad
\text{time}=4\times10^{-12}\,\text{s}
\]

\[
E_b=1\,\text{eV}:\quad
\text{rate}=10^{13}e^{-1/0.02671}=5.3\times10^{-4}\,\text{s}^{-1},\quad
\text{time}=1.9\times10^3\,\text{s}\approx32\,\text{min}
\]

\[
E_b=10\,\text{eV}:\quad
\text{rate}=10^{13}e^{-10/0.02671}\approx10^{-150}\,\text{s}^{-1},\quad
\text{time}\approx10^{150}\,\text{s}
\]

\end{document}
