\documentclass[11pt]{article}
\usepackage[margin=1in]{geometry}
\usepackage{amsmath, amssymb}
\usepackage{enumitem}
\usepackage{tikz}
\usepackage{booktabs}
\usepackage{graphicx}
\usepackage{float}
\usepackage{placeins}

\title{HW \#4}
\author{Theo Tarr}
\date{February 26, 2026}

\begin{document}
\maketitle

\section*{1. A Chemical Bond}

\textbf{(a)} With no interference, probability = sum of the individual probabilities:
\[
P = |\psi_1|^2 + |\psi_2|^2 = (0.35)^2 + (0.22)^2 = 0.1225 + 0.0484 = \boxed{0.1709}
\]

\textbf{(b)} With constructive interference, add the amplitudes before squaring:
\[
P = |\psi_1 + \psi_2|^2 = (0.35 + 0.22)^2 = (0.57)^2 = \boxed{0.3249}
\]

\textbf{(c)} The wave probability (0.32) is almost double the probability if there was no interference (0.17). This increased electron density between the two positive nuclei creates a covalent bond that holds them together. Because of the constructive interference, there is a higher probability of finding the electron between the two nuclei, which lowers the overall energy of the system and forms a stable bond.

\begin{figure}[H]
\centering
\includegraphics[width=0.5\linewidth]{0CA4DC78-8301-44DB-974D-70536716A0A0_1_201_a.jpeg}
\caption{Constructive interference increases electron density between two nuclei.}
\label{fig:q1-bonding}
\end{figure}
\FloatBarrier


\section*{2. Hydrogen Atom}

\textbf{(a)} The longest wavelength has least energy, which is the $E_3 \to E_2$ transition.

\[
E_3 = -1.5\ \text{eV}
\]
\[
E_2 = -3.4\ \text{eV}
\]

\[
\Delta E = E_3 - E_2 = -1.5 - (-3.4) = 1.9\ \text{eV}
\]
\[
\lambda = \frac{hc}{\Delta E} = \frac{1241\ \text{eV}\cdot\text{nm}}{1.9\ \text{eV}} \approx \boxed{653\ \text{nm (a red wavelength)}}
\]

\textbf{(b)}
\[
E_1 = -13.6\ \text{eV}
\]
\[
E_5 = -0.54\ \text{eV}
\]
\[
\Delta E = E_5 - E_1 = -0.54 - (-13.6) = \boxed{13.06\ \text{eV}}
\]

\textbf{(c)}
\[
\frac{P(E_5)}{P(E_1)} = e^{-\Delta E / k_B T} = \frac{1}{1000}
\]
\[
-\frac{13.056}{k_B T} = \ln(\frac{1}{1000}) = -6.908
\]
\[
k_B T = \frac{13.056}{6.908} = 1.890 \text{ eV}
\]
\[
T = \frac{1.890 \text{ eV}}{8.617 \times 10^{-5} \text{ eV/K}} = \boxed{21{,}900 \text{ K}}
\]

\section*{3. Chemical Reactions}

\textbf{(a)} The diagram shows potential energy versus position with wave functions for each state (A, B, and transition).
\begin{figure}[H]
    \centering
    \includegraphics[width=0.5\linewidth]{807F8F1B-1CA5-4785-B7BD-C9D1F89A6F56_1_201_a.jpeg}
    \caption{Double-well potential with localized states A and B and a higher-energy transition state near the barrier.}
    \label{fig:q3-doublewell}
\end{figure}
\FloatBarrier

Each well has quantized standing-wave states. The $n=1$ mode is the ground state (lowest energy) and has no internal node; higher-$n$ states have more nodes and higher energy. This is analogous to a particle in a box,
\[
E_n = \frac{n^2 h^2}{8mL^2}.
\]
So state A and state B are shown as localized low-energy standing waves in their respective wells. To convert A $\to$ B, the system must pass through the high-energy configuration near the barrier top (the transition state).

\textbf{(b)}
\[
\text{rate}=\frac{k_B T}{h}e^{-E_b/(k_B T)},\qquad \text{time}=\frac{1}{\text{rate}}.
\]
At $T=310\,\text{K}$,
\[
k_B T=(8.617\times10^{-5}\,\text{eV/K})(310\,\text{K})=0.02671\,\text{eV},
\qquad
\frac{k_B T}{h}=6.46\times10^{12}\,\text{s}^{-1}.
\]

\begin{align*}
E_b&=0.1\,\text{eV}: &
\text{rate}&=(6.46\times10^{12})e^{-0.1/0.02671}=1.53\times10^{11}\,\text{s}^{-1}, &
\text{time}&=\frac{1}{\text{rate}}\approx6.5\times10^{-12}\,\text{s}.\\[4pt]
E_b&=1\,\text{eV}: &
\text{rate}&=(6.46\times10^{12})e^{-1/0.02671}=3.42\times10^{-4}\,\text{s}^{-1}, &
\text{time}&=\frac{1}{\text{rate}}\approx2.9\times10^{3}\,\text{s}\approx48\,\text{min}.\\[4pt]
E_b&=10\,\text{eV}: &
\text{rate}&=(6.46\times10^{12})e^{-10/0.02671}\approx3\times10^{-150}\,\text{s}^{-1}, &
\text{time}&\approx3\times10^{149}\,\text{s}.
\end{align*}

\end{document}
